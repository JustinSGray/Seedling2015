\documentclass[]{aiaa-tc}

\usepackage{authblk}
%\usepackage[margin=3cm]{geometry}


\begin{document}

  %\maketitle

  \vspace{4em}
  \begin{center}
    A proposal to the NASA ARMD Team Seedling Program: 
    \vspace{2em}

    {\Huge Coupled Aero­-Structural-Propulsive­ Design of an Ultra­Light, Highly Flexible, Human Powered Aircraft}

    \vspace{2em}
    submitted to the NASA Aeronautics Research Institute by: 

    \vspace{3em}
    [Insert Lots of Logos/Picture Here]
    \vspace{3em}


  \end{center}


  \begin{tabular}{l l}
    Principal Investigator & Justin S. Gray; NASA Glenn Research Center, Propulsion Systems Analysis Branch \\ 
    & \\
    Co-Investigators & Dr. Juan Alonso; Stanford University, Department of Aeronautics and Astronautics \\
                     & Dr. Graeme Kennedy; Georgia Institute of Technology, School of Aerospace Engineering \\ 
                     & Dr. Todd Reichert; Vice President of Aerodynamics, AeroVelo Inc. \\
    & \\ 
    Team Members & Cameron Robertson; Vice President of Structures, AeroVelo Inc. \\ 
                 & Jeffrey Chin; NASA Glenn Research Center, Propulsion Systems Analysis Branch \\
                 & Kevin Reynolds; NASA Ames Research Center, Intelligent Systems Branch \\
  \end{tabular}

  \newpage

  \section{Introduction}

    The Fundamental Aeronautics Program has identified \textbf{Ultra Efficient Commercial Vehicles} as 
    a key research thrust. A number of concepts have been proposed along those lines, such as 
    the MIT D8 double­-bubble and the Boeing SUGAR­-High truss­-braced wing. These aircraft rely on 
    very high­-aspect ratio, extremely flexible wings to help achieve very low energy consumption. 
    Long endurance solar sensorcraft, being proposed by companies such as General Atomics and 
    Scaled Composites, share a similar approach to minimizing energy consumption by leveraging highly 
    flexible wings. This begs the question, \textbf{``How does aircraft design change to take full 
    advantage of highly flexible structures?''} The Vision 2030 CFD study also highlights the 
    design of highly flexible wings as a modeling grand challenge problem. That study specifically 
    highlights the need for highly multidisciplinary, tightly coupled design process with high 
    fidelity tools to address this grand challenge.

    Aircraft empty weight reduction is a major driving factor pushing designs toward highly flexible structures. 
    While lower structural weight helps to achieve lower energy consumption it also tends to result in very large 
    deflections and highly non-linear structural responses. These characteristics mean that 
    structural failure is a major design consideration.[CITATION HERE] Thus it is interesting to consider the question, 
    \textbf{``How can structural health monitoring technologies be used to enable lighter 
    weight structures?''}, to address the \textbf{Real­-Time System­-Wide Safety Assurance} thrust. 

    One of the major challenges associated with designing highly flexible wings is that the as-flown configuration 
    is dramatically different than the as-built configuration (jig shape). With modern aeroelastic techniques, 
    we can now design more conventional wings taking this into account[CITATIONS NEEDED]. However, highly flexible
    structures bring new challenges that need to be addressed. Besides the non-linear structural 
    response these wings also tend to have different load distributions from conventional wings. Many of the concepts 
    are electric aircraft with propellers spread out across the wing for distributed propulsion. This can have not only a 
    structural impact, but also an aerodynamic and controllability impact as well. These aspects push the 
    limits of todays most advanced high-fidelity aeroelastic design techniques, which currently rely on linear structural 
    models, don't account for any propulsion interactions, and don't easily handle constraints on dynamic behavior. 

    Although tools currently exist to address each one of the various modeling challenges, there is not yet a well 
    accepted method to integrating them all into a cohesive system model and then applying that model to an actual 
    aircraft design process. This proposal outlines a unique opportunity for NASA to fill this capability gap
    via an aircraft design and flight testing effort done in close collaboration with AeroVelo Inc., Georgia Institute of Technology, 
    and Stanford University. AeroVelo won the Sikorski Human Powered Helicopter Prize in 2013. Their next human powered aircraft will be 
    build to conquer the Kremer Marathon Challenge; fly a human powered aircraft 26.1 miles in under 1 hour. 
    They have invited a NASA led team of researchers to participate in this project. 

    A human powered aircraft designed for the Kremer Marathon Challenge makes an excellent test case for the 
    wide range of modeling and design challenges associated with highly flexible wings. 
    The only available power is that of the humans on board. Each one can provide about the 
    same amount of power as a cordless drill. In order to reach the speeds necessary, multiple people will be required. 
    Figure \ref{fig:aerovelo-concept} is the initial AeroVelo concept, which includes three pilots spread across the wing
    span with three propellers. Spreading the pilots out over the wing span yields a distributed load profile, 
    which must be supported by a very light weight and extremely flexible structure. Weight is not the only 
    key design consideration either. To keep aerodynamic drag low, it is critical to maintain laminar flow 
    over as much of the wing as possible. 

    \begin{figure}
        \includegraphics[width=\textwidth]{images/vsp_concept_solid}
        \caption{}
        \label{fig:aerovelo-concept}
    \end{figure}

    The combination of low speed flight, very low available power, distributed propulsion, and highly 
    flexible span-loaded structures makes this human powered aircraft an excellent analog for a wide 
    range of high altitude, long endurance drones. Hence all of the tools and methods developed under 
    this effort will be directly applicable to that class of aircraft. But by working with a human powered 
    aircraft, we are able to achieve a lower cost, rapid feedback environment to develop and validate these 
    new methods. 

    In phase one of this effort, the analysis tools and coupled design environment will be built and 
    the detailed design of the marathon aircraft will be completed. In addition, sub-scale testing will be 
    performed to validate structural models and construction techniques. In phase two of the effort
    the aircraft will be built, tested, and flown to attempt the Kremer Marathon Challenge. The aircraft will be 
    used as a flying test bed to collect further structural and aerodynamic validation data. All the data collected, 
    all of the models, and the final detailed design will be released publicly under an open-source license.

  \section{Objectives and Technical Approach}

    The overall objective of this work is to design, test, and fly a human powered aircraft, 
    capable of flying 26.1 miles in under 1 hour using state-of-the-art multidisciplinary 
    design optimization techniques. The design will be performed with fully coupled
    aerodynamic, non-linear structural, and propeller models. Working in the context of 
    a full aircraft design process, from conceptual all the way to detailed design, 
    provides a unique opportunity to apply multidisciplinary optimization methods and validate
    the results within a single project. 

    Phase one of the work will focus on the design of the aircraft. NASA Glenn Research Center, 
    NASA Ames Research Center, Stanford, and the Georgia Institute of Technology will be responsible 
    for collaboratively developing the propulsion, aerodynamic, and structural analysis 
    models needed as well as integrating them together for perform the coupled analysis. AeroVelo 
    use the models to perform the conceptual, preliminary, and detailed design of the actual marathon 
    aircraft in very close collaboration with the other partners. 

    \subsection{Model Integration (Glenn)}

    NASA Glenn will be responsible for managing the integration of the various analyses into 
    a single system model via the OpenMDAO framework. OpenMDAO provides advanced 
    capabilities for high-fidelity code coupling. These capabilities will be leveraged to combine the 
    propeller, wing aerodynamics, and structural models into a single coupled aircraft model. The aircraft 
    model will the be used as a design tool by AeroVelo to handle the actual design of the aircarft. 

    One of the key features of OpenMDAO will be its support for computing system derivatives
    analytically or semi-analytically, specifically using an adjoint formulation. The adjoint
    derivatives capability is key, because the combination of high-fidelity aerodynamics and high-fidelity 
    structural models yields a very large design space. Independent airfoils must be designed independently 
    across the entire wing. In addition, structural thickness throughout the aircraft can all be sized independently. 
    It is possible to have thousands or even 10's of thousands of design variables. This makes 


    Both TACS and 
    SU2 both provide analytic derivatives, but these need to be combined together to solve for the system level 
    coupled derivatives. OpenMDAO provides the mechanism to do this automatically, for both forward and adjoint 
    derivative formulations. 

    with support for analytic gradients which will be essential 
    for performing a design optimization at the system level. 

    \begin{figure} \centering
        \includegraphics[width=.75\textwidth]{xdsm/overall}
    \end{figure}

    \begin{itemize}
        \item System level adjoint analytic gradients for large design spaces
        \item Geometry (GeoMACH, VSP, OpenCSM)
        \item Load and displacement transfer?? 
    \end{itemize}

    

    \subsection{Aerodynamic Modeling (Stanford / Ames)}

    Stanford's SU2 CFD tool, will be used for the aerodynamic analyses of both the wing and 
    the propellers. SU2 has been successfully applied to the design of propeller systems 
    and has support for actuator disk boundary conditions that will be needed to implement 
    the coupled wing­-propeller design process. It has been demonstrated to work in the low 
    Mach conditions that will be seen by this aircraft. It also provides adjoint derivatives 
    which are fundamental to the application of state­-of-­the-­art gradient based design 
    optimization methods. Researchers at Stanford will be responsible for building for developing 
    the aerodynamic models of the wing and will collaborate with NASA Ames on developing the 
    propeller models. In addition they will provide support for integrating SU2 into the OpenMDAO 
    framework. 

    \subsubsection{Propeller Analysis (Ames)}
        \begin{itemize}
            \item SU2 can be applied using the rotating reference frame features for efficient high fidelity propeller design
            \item Lower fidelity tools may be even more efficient and could be highly accurate with good 2d airfoil data. Andrew Ning's 
            CCBlade is a good option. 
        \end{itemize}

    \subsubsection{Wing \& Fueselage Analysis (Stanford)}
        \begin{itemize}
            \item Laminar flow is a key design requirement for both the wing and the pilot farings. An inverse design approach can be employed 
            by specifying a desired pressure distribution. 
            \item TriPan is a lower fidelity option that also provides gradients and could possibly be employed to reduce computational cost. It might make sense to set things up so that SU2 and Tripan could be used interchangeably. 
        \end{itemize}

    \subsubsection{Wing-Propeller Coupling (Glenn/Stanford)}
        \begin{itemize}
            \item SU2 actuator disks for coupling
        \end{itemize}

    \subsection{Structural Analysis (GaTech)}

    The structural modeling will be performed with Toolkit for Analysis and of Composite 
    Structures (TACS). TACS has the necessary advanced capabilities such as adjoint derivatives 
    and efficient load and displacement transfer to enable coupled aero­-structural design. It also has 
    support for iso-geometric elements which will reduce the overall computational cost of the 
    structural analysis. Researchers at the Georgia Institute of Technology will be responsible for 
    building the structural models and providing support for integrating them into the OpenMDAO framework. 

    \subsection{Aircraft Design (AeroVelo)}

    AeroVelo will be responsible or the actual design and construction of the aircraft using the tools developed 
    by the other team members. They will also be responsible for building the structural test articles and designing 
    the structural validation tests. After the attempt has been made, NASA will be given access to 
    the aircraft to perform additional test flights to collect data. 

    \subsection{Wind Tunnel Testing}
        \begin{itemize}
            \item Expect airfoils to have a chord of about 2 feet. Flight speeds will be about 26 mph (40 feet/sec). Can we find a wind tunnel 
            that can test an article of that size at such slow speeds? 
            \item Another option would be to build a small test wing and fly it on a drone. Ames has the ability to rapidly prototype this kind 
            of drone and could provide actual flight tests. I suspect this would be a lower cost solution, but could 
            present measurement challenges. 
        \end{itemize}

    \subsection{Structural Testing}
        \begin{itemize}
            \item CFOSS system for structural tests?? 
            \item What kind of measurements are necessary to validate the codes? 
            \item What kind of measurements are necessary to validate the design? 
        \end{itemize}



  \section{Innovation}
    The nature of the highly flexible wing on this human powered aircraft will demand the use 
    of nonlinear structural models. A fully coupled high fidelity nonlinear structural, aerodynamic modeling method 
    will represent a significant advancement in the state­-of-­the-­art for aero­-structural design. By also including 
    propeller effects on the aerodynamics of the wing, we will be demonstrating an even larger degree of multidisciplinary 
    coupling. 

    In addition, the extremely low available power levels for human powered aircraft demand laminar flow over as much of 
    the aircraft as possible. Because of the large deformations, its important to maintain laminar flow in the 
    as-flown condition. By using the coupled aero-structural analysis we can take deformations into account in the design 
    of the airfoils. 

    Lastly, the integration of multidisciplinary design optimization into all phases of the aircraft design process represents
    a tremendous opportunity to learn how it can be best utilized. By combining MDO, testing, and actual flight into a single 
    we gain the opportunity to iterate not just on the design but also on the design process itself. The ability to adjust the 
    MDO methods to better suit the design process is a unique and highly valuable opportunity. 

  
\section{Intellectual Property Statement}
    \begin{itemize}
        \item All models will be open sourced
        \item All design will be made publicly available via a detailed design report
        \item All codes will be shared via GitHub
    \end{itemize}

\section{Virtual Meetings and Collaboration Plan}
    \begin{itemize}
        \item GitHub used to coordinate all codes and model 
        \item GoogleHangout or other VITS System for monthly meeting
        \item Two in person multi-day meetings
    \end{itemize}

\section{Milestones and Deliverables}
    \subsection{Milestones}
        \begin{itemize}
            \item geometry model for the aircraft, including structural model (Glenn/AeroVelo) 11/30/2014
            \item geometry model for the propeller (Glenn/AeroVelo) 11/30/2014
            \item FEM (TACS) Model of the wing (GATech) 
            \item Aerodynamic Model of the Wing (Stanford/Ames)
            \item Coupled Aero-Propulsive Analysis in OpenMDAO (Glenn/Stanford)
            \item Coupled Aero-Structural Analysis in OpenMDAO (Glenn/GATech/Stanford) 
            \item Coupled Aero-Structural Optimization in OpenMDAO (Glenn/GATech/Stanford)
            \item Conceptual Design for Marathon Airplane (Glenn/AeroVelo) 1/30/2015
            \item Detailed Design for Marathon Airplane (Glenn/AeroVelo) 9/30/2015
        \end{itemize}

    \subsection{Deliverables}
        \subsubsection{Year 1}
            \begin{itemize}
                \item Structural Test Articles (AeroVelo)
                \item Aerodynamic Test Articles (AerVelo)
                \item Aerodynamic Models (Stanford)
                \item Structural Models (GaTech)
                \item Final Report (All)
            \end{itemize}
        \subsubsection{Year 2}
            \begin{itemize}
                \item Additional Testing?? 
                \item Structural Dynamic Analyses?? 
                \item Flying Airplane??
            \end{itemize}

  \section{Resource Planning}
    \subsection{Test Facilities}
    \subsection{Supercomputing Resources}

  \appendix

  \section{Budget}
    Maximum available funds: 750k 
    \begin{itemize}
        \item 1 FTE Glenn - 150k 
        \item NASA Glenn WYE Support 60k 
        \item 1 FTE Ames - 150k 
        \item .25 FTE Armstrong - 40k
        \item Wind Tunnel and Structural Testing - 100k 
        \item Stanford - 125k 
        \item GATech - 125k
    \end{itemize}

  \section{Resumes \& Qualifications}


  \bibliography{references}

\end{document}